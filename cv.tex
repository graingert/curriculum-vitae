%% start of file `template.tex'.
%% Copyright 2006-2013 Xavier Danaux (xdanaux@gmail.com).
%
% This work may be distributed and/or modified under the
% conditions of the LaTeX Project Public License version 1.3c,
% available at http://www.latex-project.org/lppl/.


\documentclass[11pt,a4paper,sans]{moderncv}        % possible options include font size ('10pt', '11pt' and '12pt'), paper size ('a4paper', 'letterpaper', 'a5paper', 'legalpaper', 'executivepaper' and 'landscape') and font family ('sans' and 'roman')

% moderncv themes
\moderncvstyle{casual}                             % style options are 'casual' (default), 'classic', 'oldstyle' and 'banking'
\moderncvcolor{blue}                               % color options 'blue' (default), 'orange', 'green', 'red', 'purple', 'grey' and 'black'
%\renewcommand{\familydefault}{\sfdefault}         % to set the default font; use '\sfdefault' for the default sans serif font, '\rmdefault' for the default roman one, or any tex font name
%\nopagenumbers{}                                  % uncomment to suppress automatic page numbering for CVs longer than one page

% character encoding
\usepackage[utf8]{inputenc}                       % if you are not using xelatex ou lualatex, replace by the encoding you are using
%\usepackage{CJKutf8}                              % if you need to use CJK to typeset your resume in Chinese, Japanese or Korean

% adjust the page margins
\usepackage[scale=0.75]{geometry}
%\setlength{\hintscolumnwidth}{3cm}                % if you want to change the width of the column with the dates
%\setlength{\makecvtitlenamewidth}{10cm}           % for the 'classic' style, if you want to force the width allocated to your name and avoid line breaks. be careful though, the length is normally calculated to avoid any overlap with your personal info; use this at your own typographical risks...

\usepackage{multicol}
\usepackage[cm]{fullpage}
% personal data
\firstname{Thomas}
\familyname{Grainger}
\title{Curriculum Vitae}                               % optional, remove / comment the line if not wanted
\address{177b Brook Drive}{London SE11 4TG}{England}% optional, remove / comment the line if not wanted; the "postcode city" and and "country" arguments can be omitted or provided empty
\mobile{+44~(7733)~933~161}                          % optional, remove / comment the line if not wanted
%\phone{+2~(345)~678~901}                           % optional, remove / comment the line if not wanted
%\fax{+3~(456)~789~012}                             % optional, remove / comment the line if not wanted
\email{mail@graingert.co.uk}                               % optional, remove / comment the line if not wanted
%\homepage{graingert.co.uk}                         % optional, remove / comment the line if not wanted
%\extrainfo{additional information}                 % optional, remove / comment the line if not wanted
\photo[64pt][0.4pt]{picture}                       % optional, remove / comment the line if not wanted; '64pt' is the height the picture must be resized to, 0.4pt is the thickness of the frame around it (put it to 0pt for no frame) and 'picture' is the name of the picture file
%\quote{Some quote}                                 % optional, remove / comment the line if not wanted

% to show numerical labels in the bibliography (default is to show no labels); only useful if you make citations in your resume
%\makeatletter
%\renewcommand*{\bibliographyitemlabel}{\@biblabel{\arabic{enumiv}}}
%\makeatother
%\renewcommand*{\bibliographyitemlabel}{[\arabic{enumiv}]}% CONSIDER REPLACING THE ABOVE BY THIS

% bibliography with mutiple entries
%\usepackage{multibib}
%\newcites{book,misc}{{Books},{Others}}
%----------------------------------------------------------------------------------
%            content
%----------------------------------------------------------------------------------
\begin{document}
%\begin{CJK*}{UTF8}{gbsn}                          % to typeset your resume in Chinese using CJK
%-----       resume       ---------------------------------------------------------
\makecvtitle

\begin{minipage}{\linewidth}
\section{Education}
\cventry{2009--2013}{MEng Computer Science}{University of Southampton}{2.1}{}{}
\cventry{2007--2009}{A Level}{Peter Symonds College}{Winchester}{}{
  \begin{multicols}{2}
    \begin{itemize}
      \item A2 Maths: A
      \item A2 Further Maths: A
      \item A2 Physics: A
      \item A2 Chemistry: B
      \item AS Biology: B
    \end{itemize}
  \end{multicols}
}
\cventry{2003--2007}{10 GCSEs A*-C Including English, Maths and Science}{The Westgate School}{Winchester}{}{}
\end{minipage}


\section{Fourth year dissertation}
\cvitem{Title}{\emph{Remaining Anonymous when using the Bitcoin Protocol}}
\cvitem{Supervisor}{Dr. Tim Chown}
\cvitem{Description}{Investigates the methods used for de-anonymizing or revealing
the identity of users of the Bitcoin network and how these methods can be avoided
as well as proposing a novel automated solution to the issue of such de-anonymizing
techniques.}

\begin{minipage}{\linewidth}
\vspace{3ex}
\section{Experience}
\cventry{July 2013 - Current}{Python Developer}{Hogarth Worldwide, Technology Services Department}{}{}{
  Worked as member of a nine person agile development team on a workflow and
  asset management web application from inception to final delivery. Alongside
  Django development, took responsibility for the design and implementation
  of a large integration project with legacy systems and assisted in the
  transition to an XHR (AJAX) driven single-page application. Worked with
  DevOps to package and deploy the application using SaltStack configuration
  management and VMware vSphere. Introduced GitHub pull-request driven
  code-review. Reduced a multi-hour Jenkins build queue to zero. Ensured
  that the product was delivered on schedule despite extensive rewrites of the
  application due to requirement changes.
}
\cventry{Summer 2012}{Research Assistant}{University of Southampton, Web and Internet Science Research Group}{}{}{
  Investigated University IT security and enhanced a PHP prototype web application related to the security of the eduroam world-wide roaming access service.
  Specifically security vulnerabilities such as XSS and CSRF were removed and the application was rearchitected to use front-end rednering with AngularJS communicating to an HTTP API backend.
  Part of this process involved porting the code to Django, with raw SQL being converted to the equivalent Django ORM queries.
}
\cventry{Summer 2010 and 2011}{Formal Test}{IBM Storage Subsystems Group}{Hursley}{}{
  Developed an application to parse and store RAID performance test results and Django website for analysing those results: ``Perfweb''.
  Formal test for the IBM Storwize v7000, announced in October 2010 requiring the use of the \href{http://www.lecroy.com/ProtocolAnalyzer/ProtocolOverview.aspx?seriesid=147}{LeCroy SAS Sierra M6-4} Protocol Analyser/Jammer.
}
\cventry{Summers 2006--2008}{Developer}{IBM Storage Subsystems Group}{Hursley}{}{
  Multiple short work experience summers developing ``pl\_scsi'' a Cluster based SCSI command injector in C and a web based error log decoder in Perl. Both of these projects required use of the Finisar Fibre Channel Protocol Analyser/Jammer.
}
\end{minipage}

\begin{minipage}{\linewidth}
\vspace{3ex}
\section{Programming Languages}
  \cventry{Python}{Expert}{}{}{}{With contributions to open source projects such as Gunicorn, Celery and Eventlet. Maintainer of \href{http://github.com/graingert/python-clamd}{python-clamd}.}
  \cventry{JavaScript}{Competent}{}{}{}{From jQuery to AngularJS, with open source contributions. Maintainer of the NPM package slimerjs.}
  \cventry{Java}{Competent}{}{}{}{Including Swing, Threading and RMI.}
  \cventry{C/C++}{Some Experience}{}{}{}{Simple allocator implementation, OpenGL graphics project.}
  \cventry{PHP, Perl, Ruby}{Some Experience}{}{}{}{Simple web applications, security patches to existing systems.}
\end{minipage}

\begin{minipage}{\linewidth}
\vspace{3ex}
\section{Computer Skills}
\cvitem{Source Control}{Git (advanced), Subversion and \href{https://en.wikipedia.org/wiki/CMVC}{CMVC}.}
\cvitem{Testing}{Developed using TDD with automated test tools including nose, py.test and tox. Advanced use and configation of Jenkins jobs, slaves and executors.}
\cvitem{Web Technologies}{HTML technologies such as WebSockets, Semantic Web technologies such as RDF and SPARQL.}
\cvitem{Frameworks}{Django, Flask and Rails.}
\cvitem{DevOps}{VMware vSphere, Ubuntu 12.04.4 Linux, use of and contributions to SaltStack}
\cvitem{Distributed Computing}{Celery, RabbitMQ.}
\cvitem{Databases}{MySQL/MariaDB, PostgreSQL and Redis.}
\end{minipage}

\begin{minipage}{\linewidth}
\vspace{3ex}
\section{Interests \& Side Projects}
\cvitem{Humanism}{
  Committee member of the Southampton Atheist Society between 2010--2013 and
  since moving to London now regularly attend the Central London Humanist
  Group.
}
\cvitem{IPv6}{
  Deployed the next-generation Internet Protocol, IPv6, at home and consequently
  earned \href{https://ipv6.he.net/certification/}{Hurricane Electric IPv6
  Certification}.
}
\cvitem{Cyber security Workshop}{
  Wrote and delivered a workshop teaching Wi-Fi, Web and Network penetration
  testing as part of the University of Southampton's UK-China Cyber Security workshop.
}
\end{minipage}

\begin{minipage}{\linewidth}
\vspace{3ex}
\section{References}
\begin{cvcolumns}
  \cvcolumn{IBM Internships}{
    Jon Short\\
    IBM Storage Subsystems Group, Hursley\\
    SAN Volume Controller, Regression Test Team Leader\\
  }
  \cvcolumn{University of Southampton Internship}{
    \href{http://id.ecs.soton.ac.uk/person/446}{Dr. Tim Chown}\\
    ECS, Faculty of Physical and Applied Sciences\\
    University of Southampton\\
  }
\end{cvcolumns}
\end{minipage}


% Publications from a BibTeX file without multibib
%  for numerical labels: \renewcommand{\bibliographyitemlabel}{\@biblabel{\arabic{enumiv}}}% CONSIDER MERGING WITH PREAMBLE PART
%  to redefine the heading string ("Publications"): \renewcommand{\refname}{Articles}
\nocite{*}
\bibliographystyle{plain}
\bibliography{publications}                        % 'publications' is the name of a BibTeX file

% Publications from a BibTeX file using the multibib package
%\section{Publications}
%\nocitebook{book1,book2}
%\bibliographystylebook{plain}
%\bibliographybook{publications}                   % 'publications' is the name of a BibTeX file
%\nocitemisc{misc1,misc2,misc3}
%\bibliographystylemisc{plain}
%\bibliographymisc{publications}                   % 'publications' is the name of a BibTeX file

%\clearpage\end{CJK*}                              % if you are typesetting your resume in Chinese using CJK; the \clearpage is required for fancyhdr to work correctly with CJK, though it kills the page numbering by making \lastpage undefined
\end{document}


%% end of file `template.tex'.
